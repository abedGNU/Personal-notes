%----------------------------------------------------------------------------------------
%	CHAPTER 
%----------------------------------------------------------------------------------------

\chapterimage{chapter_head_2.pdf} % Chapter heading image
\chapter{ROS tutorials}

%----------------------------------------------------------------------------------------
%
%----------------------------------------------------------------------------------------
\section{Install ROS}

%----------------------------------------------------------------------------------------
%
%----------------------------------------------------------------------------------------
\section{Configuring Your ROS Environment}

%------------------
%
%------------------
\subsection{Managing Your Environment}

%------------------
%
%------------------
\subsection{Create a ROS Workspace}
Let's create and build a catkin workspace:
\begin{lstlisting}
	mkdir -p ~/catkin_ws/src
	cd ~/catkin_ws/
	catkin_make
\end{lstlisting}

The catkin\_make command is a convenience tool for working with catkin workspaces. Running it the first time in your workspace, it will create a CMakeLists.txt link in your 'src' folder. Additionally, if you look in your current directory you should now have a 'build' and 'devel' folder. Inside the 'devel' folder you can see that there are now several setup.*sh files. Sourcing any of these files will overlay this workspace on top of your environment. To understand more about this see the general catkin documentation: catkin. Before continuing source your new setup.*sh file:

\begin{lstlisting}
	source devel/setup.bash
\end{lstlisting}

To make sure your workspace is properly overlayed by the setup script, make sure ROS\_PACKAGE\_PATH environment variable includes the directory you're in.
\begin{lstlisting}
	echo \$ROS\_PACKAGE\_PATH
	// /home/youruser/catkin_ws/src:/opt/ros/kinetic/share
\end{lstlisting}

%------------------
%
%------------------
